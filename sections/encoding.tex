\subsubsection{Selection of Bitrates}
Video Multi-Method Assessment Fusion (VMAF) is a full reference metric for estimating human perception of video quality \cite{lin2014:fvqa}.

\begin{figure}[!t]
	\centering
	\includegraphics[width=3.5in]{vmaf_bitrates}
	\caption{VMAF scores for 25 different bitrates at 3 resolutions}
	\label{fig:vmaf:bitrates}
\end{figure}

To estimate relevant HEVC bitrates for our source content we sample the VMAF scores for different resolutions. The reference sequences are resampled to a fixed 50 frames per seconds to avoid frame rate differences, while the distorted sequences are downsampled, encoded with CBR rate control and upsampled to 4k again using lanczos resampling. Both presets use 4:2:0 chroma subsampling to be close to the typical use-case of webvideo. The resulting VMAF scores show an overlap between different resolutions as seen in Fig. \ref{fig:vmaf:bitrates} and the final encoding bitrates are chosen near those intersections.

\begin{figure}[!t]
	\centering
	\includegraphics[width=3.5in]{vmaf_final}
	\caption{VMAF scores of encoded videos}
	\label{fig:vmaf:encoded}
\end{figure}

\subsubsection{Encoding Presets}
Two different presets are used for the sequence encodings. The first is a simple CBR-encoding and the second a 2-pass encoding with a Q-CTRL pass followed by a B-CTRL pass. Every sequence is encoded with both presets at the 3 resolutions and 3 bitrates. The resulting VMAF-scores for the encoded sequences can be seen in figure \ref{fig:vmaf:encoded}.

\begin{figure}[!t]
	\centering
	\includegraphics[width=3in]{automation}
	\caption{Automated processing and encoding workflow for providing }
	\label{fig:automation}
\end{figure}

\subsubsection{Encoding Automation}
We automate the whole process for downloading, preprocessing and encoding the source videos using pydoit \cite{web:pydoit}.

The process is illustrated in figure \ref{fig:automation} and starts with the source preparation (Blue box). After downloading the sequences (1) they are cut to 10 seconds length and saved as ProRes HQ with UHD-1 resolution (2). Additionally, h.264 previews are generated at a lower resolution of 1440p to allow reviewing of sequences on slower devices.



