Broader access to online content \cite{forecast2016cisco} combined with and increased diversity of available content has led to a saturation of internet traffic \cite{sandvine2010global}. This issue has been mitigated through changes in how content is delivered, these changes include more intelligent delivery approaches such as using content delivery networks (CDN)\cite{farber2003internet} and Adaptive Bitrate (ABR) streaming \cite{brueck2010apparatus}. These new approaches go hand in hand with a need for improvements in the way content is prepared before delivery. The two aspects addressed in this paper are the link between encoding parameters and user-perceived video quality as well as correlation between predicted video quality and perceived video quality \textbf{needs reformulation}.

This paper aims at finding correlation between predicted quality and user ratings based on three different encoding parameters: resolution, bitrate and two HEVC presets

Several data has been collected in addition to user ratings, these data include: feedback questions, rating behaviour. This additional data has been collected in order to enable a deeper analysis of the subjective test results.

Achieving reproducible results being a strong focus of this research project the different steps of our work have largely been automated \textbf{needs reformulation}.

This report describes the different steps of our work in the following order: first content selection is described, then the encoding stuff, then the test design and finally how we put the guacamole together! \textbf{lolmdr}