Broader access to online media content, especially on mobile devices \cite{forecast2016cisco}, combined with an growing diversity of content has led to an increased saturation of internet backbone networks \cite{sandvine2010global}. More dynamic delivery approaches like HTTP Adaptive Streaming (\textit{HAS})\cite{seufert:2015:hassurvey} are applied to increase reliability for changing bandwidth environments, while still maintaining a high quality of experience for the consumer. This requires however, that the media content is preprocessed before delivery with video encoding and segmentation. The optimum choice of encoding parameters depends on a number of factors and is often hard to determine as generalized quality models for Adaptive Streaming are still in development \cite{raake:2017:hasqualitymodel} and not yet generally available. 

This paper aims at finding a link between predicted quality and user ratings based on the Video Multi-Method Assessment Fusion (\textit{VMAF}) metric \cite{lin2013:mmf,lin2014:fvqa} and a subjective quality experiment following ITU P.910 recommendations \cite{rec1998p}. We analyze encodings of six different 10-second sequences at varying resolutions up to \textit{UHD-1} \cite{dvb:2015:uhd1}, with three bitrates per resolution, and compare them with ratings from the perceptual test based on two different encoding presets (a "na\"{\i}ve" approach and an "expert" one).

We describe our selection of the source sequences, the encoding process and the setup for the perceptual test. As achieving reproducible results is a strong focus of this research project the preprocessing and video encoding steps have been largely automated.

Several data has been collected in addition to user ratings, such as feedback questions and rating behavior. This additional data has been collected in order to enable a deeper analysis of the subjective test results.